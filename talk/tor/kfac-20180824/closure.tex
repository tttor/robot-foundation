\section{Closure}
\frame{\tableofcontents[currentsection, hideothersubsections]}

\begin{frame}
\frametitle{Closure}
{\footnotesize

Skipped parts:
\begin{itemize}
\item Interpretations of the approximation
\item Structured Inverses (Pourahmadi, 2011)
\item Invariance properties and the relationship to whitening and centering
\item The efficiency improvements, eg damping, momentum, etc
\item Related works: Hessian-Free Opt, Diagonal Gauss-Newton, Gauss-Newton CG, etc
\end{itemize}

Other resources on KFAC:
\begin{itemize}
\item \url{https://arxiv.org/abs/1503.05671} (58-page version)
\item \url{http://www.cs.toronto.edu/~jmartens/docs/KFAC3-MATLAB.zip}
\item \url{https://github.com/tensorflow/kfac} (several in Pytorch, but not official)
\end{itemize}

Follow up:
\begin{itemize}
\item Roger Grosse et al: A Kronecker-factored approximate Fisher matrix for convolution layers. ICML 2016.
\item Yuhuai Wu, et al: Scalable trust-region method for deep reinforcement learning using Kronecker-factored approximation. NIPS 2017.
\item Jimmy Ba, et al: Distributed second-order optimization using Kronecker-factored approximations. ICLR 2017.
\end{itemize}

}
\end{frame}

